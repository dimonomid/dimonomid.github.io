T\+Neo changelog\hypertarget{changelog_changelog_current}{}\section{Current development version (\+B\+E\+T\+A)}\label{changelog_changelog_current}

\begin{DoxyItemize}
\item Fixed build without {\ttfamily \hyperlink{tn__cfg__default_8h_a7ce674edab5345c4b8d2ec251eea18eb}{T\+N\+\_\+\+U\+S\+E\+\_\+\+M\+U\+T\+E\+X\+ES}} or {\ttfamily \hyperlink{tn__cfg__default_8h_a6ed3ec7b0d4338e7f60dde86b7ea5fa4}{T\+N\+\_\+\+M\+U\+T\+E\+X\+\_\+\+D\+E\+A\+D\+L\+O\+C\+K\+\_\+\+D\+E\+T\+E\+CT}}
\item Added support of {\ttfamily -\/pedantic} mode for Cortex-\/M architectures
\end{DoxyItemize}\hypertarget{changelog_changelog_v1_08}{}\section{v1.\+08}\label{changelog_changelog_v1_08}
Release date\+: 2017-\/02-\/25


\begin{DoxyItemize}
\item {\bfseries Timers A\+PI changed}\+: now, timer callback {\ttfamily \hyperlink{tn__timer_8h_a3139b6e571f7bd3d304a7bb87b5b2459}{T\+N\+\_\+\+Timer\+Func}} is called with global interrupts enabled.
\item Fix for pic24/dspic\+: previously, initial value of P\+S\+V\+P\+AG for new tasks was always 0, but it is not necessarily the case. This might cause troubles with constants in program space. Now, when initializing stack for new task, current value of P\+S\+V\+P\+AG is used.
\item Fixed round robin\+:
\begin{DoxyItemize}
\item Even though the tasks were flipped in the runnable tasks queue, the actual context switch wasn\textquotesingle{}t performed;
\item Tasks were switched with a requested period + 1 tick.
\end{DoxyItemize}
\item Added an option {\ttfamily \hyperlink{tn__cfg__default_8h_a55cf98b6d6de554b433f497f62c2b363}{T\+N\+\_\+\+F\+O\+R\+C\+E\+D\+\_\+\+I\+N\+L\+I\+NE}}
\item Added an option {\ttfamily \hyperlink{tn__cfg__default_8h_a50a44e397e61bf00e33ebb25ac417c56}{T\+N\+\_\+\+M\+A\+X\+\_\+\+I\+N\+L\+I\+NE}}
\end{DoxyItemize}\hypertarget{changelog_changelog_v1_07}{}\section{v1.\+07}\label{changelog_changelog_v1_07}
Release date\+: 2015-\/03-\/17


\begin{DoxyItemize}
\item Fix\+: project was unable to build with {\ttfamily \hyperlink{tn__cfg__default_8h_a1f197294df3276fec431930545acafd5}{T\+N\+\_\+\+C\+H\+E\+C\+K\+\_\+\+P\+A\+R\+AM}} set to 0
\item Fix\+: Cortex-\/\+M0/\+M0+ port didn\textquotesingle{}t work if there is some on-\/context-\/switch handler ({\ttfamily \hyperlink{tn__cfg__default_8h_a49a546b18cc1f75b51d4cf8b290634dd}{T\+N\+\_\+\+P\+R\+O\+F\+I\+L\+ER}} or {\ttfamily \hyperlink{tn__cfg__default_8h_ac6a9bbac3b3b25d9b5bc8c21d2e09955}{T\+N\+\_\+\+S\+T\+A\+C\+K\+\_\+\+O\+V\+E\+R\+F\+L\+O\+W\+\_\+\+C\+H\+E\+CK}})
\item Added support of C++ compiler (experimental)
\item Added an option {\ttfamily \hyperlink{tn__cfg__default_8h_aab948a5a8178594322f364800e0e53ee}{T\+N\+\_\+\+I\+N\+I\+T\+\_\+\+I\+N\+T\+E\+R\+R\+U\+P\+T\+\_\+\+S\+T\+A\+C\+K\+\_\+\+S\+P\+A\+CE}}
\item Added services to get count of free and used memory blocks ({\ttfamily \hyperlink{tn__fmem_8h_ae8b80e0a0a8b07e11acbc4f15f3ef32d}{tn\+\_\+fmem\+\_\+free\+\_\+blocks\+\_\+cnt\+\_\+get()}} / {\ttfamily \hyperlink{tn__fmem_8h_a2f23f2cdb3e6f3f2a1c1f9ebc4e6e8cb}{tn\+\_\+fmem\+\_\+used\+\_\+blocks\+\_\+cnt\+\_\+get()}}) and items in the queue ({\ttfamily \hyperlink{tn__dqueue_8h_af541f16598e13f980ddc2482e6af6044}{tn\+\_\+queue\+\_\+free\+\_\+items\+\_\+cnt\+\_\+get()}} / {\ttfamily \hyperlink{tn__dqueue_8h_aa17fd667cab2e4a41d9e89b8cba603cd}{tn\+\_\+queue\+\_\+used\+\_\+items\+\_\+cnt\+\_\+get()}}).
\item Removed some checks from {\ttfamily \hyperlink{tn__sys_8h_aa44d297639e0520420890ef2bb7e2c1c}{tn\+\_\+tick\+\_\+int\+\_\+processing()}}, since they aren\textquotesingle{}t too useful there, but they add overhead (See \href{https://bitbucket.org/dfrank/tneokernel/issue/2/system-clock-service-routine-could-be-more}{\tt bitbucket issue \#2} )
\item Added functions for disabling/enabling scheduler\+: {\ttfamily \hyperlink{tn__sys_8h_af1502619506a9c92eb954e45ff0f461b}{tn\+\_\+sched\+\_\+dis\+\_\+save()}} / {\ttfamily \hyperlink{tn__sys_8h_a4cd5c9113872b2008184c567907653bd}{tn\+\_\+sched\+\_\+restore()}}.
\item Id fields of objects ({\ttfamily enum \hyperlink{tn__common_8h_ae779dd1f6735f6e139fb70acd004d976}{T\+N\+\_\+\+Obj\+Id}}) are moved to the beginning of object structures, to make memory corruptions detected earlier.
\item Idle task is now created with name \char`\"{}\+Idle\char`\"{} specified.
\end{DoxyItemize}\hypertarget{changelog_changelog_v1_06}{}\section{v1.\+06}\label{changelog_changelog_v1_06}
Release date\+: 2015-\/01-\/02.


\begin{DoxyItemize}
\item Cortex-\/\+M0/\+M0+/\+M1/\+M3/\+M4/\+M4F architectures are now supported.
\begin{DoxyItemize}
\item The following compilers are tested\+:
\begin{DoxyItemize}
\item A\+R\+M\+CC (Keil Real\+View)
\item G\+CC
\end{DoxyItemize}
\item Should work but not tested carefully\+:
\begin{DoxyItemize}
\item clang
\item I\+AR
\end{DoxyItemize}
\end{DoxyItemize}
\item Software task stack overflow check (optionally), see {\ttfamily \hyperlink{tn__cfg__default_8h_ac6a9bbac3b3b25d9b5bc8c21d2e09955}{T\+N\+\_\+\+S\+T\+A\+C\+K\+\_\+\+O\+V\+E\+R\+F\+L\+O\+W\+\_\+\+C\+H\+E\+CK}} for details.
\item Dynamic tick, or {\itshape tickless} (optionally)\+: refer to the page \hyperlink{time_ticks}{Time ticks} for details.
\item Profiler (optionally)\+: allows to see how much time task was running, how much time it was waiting and for what it was waiting, and so on. Refer to the documentation of {\ttfamily struct \hyperlink{structTN__TaskTiming}{T\+N\+\_\+\+Task\+Timing}} for details.
\item Old T\+N\+Kernel events compatibility mode, see {\ttfamily \hyperlink{tn__cfg__default_8h_ac61d5f6a716cdcab205a2c8afbde4242}{T\+N\+\_\+\+O\+L\+D\+\_\+\+E\+V\+E\+N\+T\+\_\+\+A\+PI}} for details.
\item Event groups\+: added {\ttfamily \hyperlink{tn__eventgrp_8h_a9d42ee61ae8da342f1cd6440b7e54bbdaf45098235d31f72a2b09e30792686573}{T\+N\+\_\+\+E\+V\+E\+N\+T\+G\+R\+P\+\_\+\+W\+M\+O\+D\+E\+\_\+\+A\+U\+T\+O\+C\+LR}} flag which allows to clear event bits atomically when task successfully finishes waiting for these event bits.
\item P\+I\+C24/ds\+P\+IC\+: little optimization\+: ffs (find-\/first-\/set bit) is implemented in an efficient P\+I\+C24/ds\+P\+I\+C-\/specific way, so finding next task to run now works a bit faster.
\item Added run-\/time check which ensures that build-\/time options for the kernel match ones for the application. For details, refer to the option {\ttfamily \hyperlink{tn__cfg__default_8h_aacdc913eb66492cf69cf02a5de73578e}{T\+N\+\_\+\+C\+H\+E\+C\+K\+\_\+\+B\+U\+I\+L\+D\+\_\+\+C\+FG}}. {\bfseries Note}\+: in your existing project that uses T\+Neo as a separate library, you need either\+:
\begin{DoxyItemize}
\item Include the file {\ttfamily $<$tneo\+\_\+path$>$/src/tn\+\_\+app\+\_\+check.c} to the application project (recommended);
\item In your {\ttfamily tn\+\_\+cfg.\+h} file, set {\ttfamily \hyperlink{tn__cfg__default_8h_aacdc913eb66492cf69cf02a5de73578e}{T\+N\+\_\+\+C\+H\+E\+C\+K\+\_\+\+B\+U\+I\+L\+D\+\_\+\+C\+FG}} to {\ttfamily 0} and rebuild the kernel with the new configuration (not recommended).
\end{DoxyItemize}

But if you build T\+Neo together with the application, this option is useless, so then just set {\ttfamily \hyperlink{tn__cfg__default_8h_aacdc913eb66492cf69cf02a5de73578e}{T\+N\+\_\+\+C\+H\+E\+C\+K\+\_\+\+B\+U\+I\+L\+D\+\_\+\+C\+FG}} to {\ttfamily 0}.
\item M\+P\+L\+A\+BX projects for P\+I\+C32 and P\+I\+C24/ds\+P\+IC moved to {\ttfamily lib\+\_\+project} directory. If you use these library projects from the repository directly in your application, you need to modify path to the library project in your application project.
\item The project\textquotesingle{}s name is shortened to {\bfseries T\+Neo}.
\end{DoxyItemize}\hypertarget{changelog_changelog_v1_04}{}\section{v1.\+04}\label{changelog_changelog_v1_04}
Release date\+: 2014-\/11-\/04.


\begin{DoxyItemize}
\item Added P\+I\+C24/ds\+P\+IC support, refer to the page \hyperlink{arch_specific_pic24_details}{P\+I\+C24/ds\+P\+IC port details};
\item P\+I\+C32\+: Core Software Interrupt is now handled by the kernel completely, application shouldn\textquotesingle{}t set it up anymore. Refer to the page \hyperlink{arch_specific_pic32_details}{P\+I\+C32 port details}.
\item Refactor\+: the following symbols\+: {\ttfamily N\+U\+LL}, {\ttfamily B\+O\+OL}, {\ttfamily T\+R\+UE}, {\ttfamily F\+A\+L\+SE} now have the {\ttfamily T\+N\+\_\+} prefix\+: {\ttfamily \hyperlink{tn__common_8h_adbf9c425147511997eb1396c4afeac40}{T\+N\+\_\+\+N\+U\+LL}}, {\ttfamily \hyperlink{tn__common_8h_a9f76389d1506addfc7542f54e484a92c}{T\+N\+\_\+\+B\+O\+OL}}, {\ttfamily \hyperlink{tn__common_8h_a87ed178a95496e2f19aa670adc4886b3}{T\+N\+\_\+\+T\+R\+UE}}, {\ttfamily \hyperlink{tn__common_8h_aa32a38a7d2e3b7983e53a61ab05417ee}{T\+N\+\_\+\+F\+A\+L\+SE}}. This is because non-\/prefixed symbols may be defined by some other program module, which leads to conflicts. The easiest and robust way is to add unique prefix.
\item Refactor\+: P\+I\+C32 M\+P\+L\+A\+BX project renamed from {\ttfamily tneo.\+X} to {\ttfamily tneo\+\_\+pic32.\+X}.
\item Refactor\+: P\+I\+C32 I\+SR macros renamed\+: {\ttfamily \hyperlink{tn__arch__pic32_8h_afde61cfabbeed1562ea2135fea8e919b}{tn\+\_\+soft\+\_\+isr()}} -\/$>$ {\ttfamily \hyperlink{tn__arch__pic32_8h_a02d853d8d573f928fb8da65ef0c2bc8e}{tn\+\_\+p32\+\_\+soft\+\_\+isr()}}, {\ttfamily \hyperlink{tn__arch__pic32_8h_a714962dd7eb6b94feee7f4d769b601c0}{tn\+\_\+srs\+\_\+isr()}} -\/$>$ {\ttfamily \hyperlink{tn__arch__pic32_8h_a523bb667617e6bb6f68a8f85855030a5}{tn\+\_\+p32\+\_\+srs\+\_\+isr()}}. It is much easier to maintain documentation for different macros if they have different names; more, the signature of these macros is architecture-\/dependent. Old names are also available for backward compatibility.
\end{DoxyItemize}\hypertarget{changelog_changelog_v1_03}{}\section{v1.\+03}\label{changelog_changelog_v1_03}
Release date\+: 2014-\/10-\/20.


\begin{DoxyItemize}
\item Added a capability to connect an \hyperlink{tn__eventgrp_8h}{event group} to other system objects, particularly to the \hyperlink{tn__dqueue_8h}{queue}. This offers a way to wait for messages from multiple queues with just a single system call. Refer to the section \hyperlink{tn__eventgrp_8h_eventgrp_connect}{Connecting an event group to other system objects} for details. Example project that demonstrates that technique is also available\+: {\ttfamily examples/queue\+\_\+eventgrp\+\_\+conn}.
\item P\+I\+C32 Interrupts\+: this isn\textquotesingle{}t a mandatory anymore to use kernel-\/provided macros {\ttfamily \hyperlink{tn__arch__pic32_8h_a02d853d8d573f928fb8da65ef0c2bc8e}{tn\+\_\+p32\+\_\+soft\+\_\+isr()}} or {\ttfamily \hyperlink{tn__arch__pic32_8h_a523bb667617e6bb6f68a8f85855030a5}{tn\+\_\+p32\+\_\+srs\+\_\+isr()}}\+: interrupts can be defined with standard way too\+: this particular I\+SR will use task\textquotesingle{}s stack instead of interrupt stack, therefore it takes much more R\+AM and works a bit faster. There are no additional constraints on I\+SR defined without kernel-\/provided macro\+: in either I\+SR, you can call the same set of kernel services. Refer to the page \hyperlink{interrupts}{Interrupts} for details.
\item Priority 0 is now allowed to use by application (in the original T\+N\+Kernel, it was reserved for the timer task, but T\+Neo \hyperlink{tnkernel_diff_tnkernel_diff_timer_task}{does not have timer task})
\item Application is now available to specify how many priority levels does it need for, it helps to save a bit of R\+AM. For details, refer to {\ttfamily \hyperlink{tn__cfg__default_8h_aad74a059c61567c68a1e9067ab47a256}{T\+N\+\_\+\+P\+R\+I\+O\+R\+I\+T\+I\+E\+S\+\_\+\+C\+NT}}.
\item Added example project {\ttfamily examples/queue} that demonstrates the pattern on how to use \hyperlink{tn__dqueue_8h}{queue} together with \hyperlink{tn__fmem_8h}{fixed memory pool} effectively.
\end{DoxyItemize}\hypertarget{changelog_changelog_v1_02}{}\section{v1.\+02}\label{changelog_changelog_v1_02}
Release date\+: 2014-\/10-\/14.


\begin{DoxyItemize}
\item Added \hyperlink{tn__timer_8h}{timers}\+: kernel objects that are used to ask the kernel to call some user-\/provided function at a particular time in the future;
\item Removed {\ttfamily tn\+\_\+sys\+\_\+time\+\_\+set()} function, because now T\+Neo uses internal system tick count for timers, and modifying system tick counter by user is a {\itshape really} bad idea.
\end{DoxyItemize}\hypertarget{changelog_changelog_v1_01}{}\section{v1.\+01}\label{changelog_changelog_v1_01}
Release date\+: 2014-\/10-\/09.


\begin{DoxyItemize}
\item {\bfseries F\+IX\+:} {\ttfamily \hyperlink{tn__dqueue_8h_ab47ed49af7dffc5a71eaabd25422d0e4}{tn\+\_\+queue\+\_\+receive()}} and {\ttfamily \hyperlink{tn__fmem_8h_a2ecd094041dbd0e92d61b852b7952444}{tn\+\_\+fmem\+\_\+get()}} \+: if non-\/zero {\ttfamily timeout} is in effect, then {\ttfamily \hyperlink{tn__common_8h_aa43bd3da1ad4c1e61224b5f23b369876a5b4d73fde6b5d1c9579c02e6aafce1fb}{T\+N\+\_\+\+R\+C\+\_\+\+T\+I\+M\+E\+O\+UT}} is returned, but user-\/provided pointer is altered anyway (some garbage data is written there). This bug was inherited from T\+N\+Kernel.
\item Added {\ttfamily \hyperlink{tn__tasks_8h_a18408d825c0dab03511f3aaeeb3ffbb3}{tn\+\_\+task\+\_\+state\+\_\+get()}}
\item {\ttfamily \hyperlink{tn__oldsymbols_8h_abdc5c428590ff525cdb566da613015ce}{tn\+\_\+sem\+\_\+acquire()}} and friends are renamed to {\ttfamily \hyperlink{tn__sem_8h_acfed2a27719c87638d2eb6edfdeea150}{tn\+\_\+sem\+\_\+wait()}} and friends. More on this read \hyperlink{tnkernel_diff_tnkernel_diff_api_rename_sem}{here}. Old name is still available through {\ttfamily \hyperlink{tn__oldsymbols_8h}{tn\+\_\+oldsymbols.\+h}}.
\end{DoxyItemize}\hypertarget{changelog_changelog_v1_0}{}\section{v1.\+0}\label{changelog_changelog_v1_0}
Release date\+: 2014-\/10-\/01.


\begin{DoxyItemize}
\item Initial stable version of T\+Neo. Lots of work done\+: thorough review and re-\/implementation of T\+N\+Kernel 2.\+7, implemented detailed unit tests, and so on. 
\end{DoxyItemize}