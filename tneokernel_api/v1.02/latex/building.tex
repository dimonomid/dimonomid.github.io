Some notes on building the project\hypertarget{building_configuration_file}{}\section{Configuration file}\label{building_configuration_file}
T\+Neo\+Kernel is intended to be built as a library, separately from main project (although nothing prevents you from bundling things together, if you want to).

There are various options available which affects A\+P\+I and behavior of the kernel. But these options are specific for particular project, and aren't related to the kernel itself, so we need to keep them separately.

To this end, file {\ttfamily \hyperlink{tn_8h}{tn.\+h}} (the main kernel header file) includes {\ttfamily tn\+\_\+cfg.\+h}, which isn't included in the repository (even more, it is added to {\ttfamily .hgignore} list actually). Instead, default configuration file {\ttfamily \hyperlink{tn__cfg__default_8h}{tn\+\_\+cfg\+\_\+default.\+h}} is provided, and when you just cloned the repository, you might want to copy it as {\ttfamily tn\+\_\+cfg.\+h}. Or even better, if your filesystem supports symbolic links, copy it somewhere to your main project's directory (so that you can add it to your V\+C\+S there), and create symlink to it named {\ttfamily tn\+\_\+cfg.\+h} in the T\+Neo\+Kernel source directory, like this\+: \begin{DoxyVerb}$ cd /path/to/tneokernel/src
$ cp ./tn_cfg_default.h /path/to/main/project/lib_cfg/tn_cfg.h
$ ln -s /path/to/main/project/lib_cfg/tn_cfg.h ./tn_cfg.h
\end{DoxyVerb}


Default configuration file contains detailed comments, so you can read them and configure behavior as you like.\hypertarget{building_building_pic32}{}\section{P\+I\+C32 port\+: M\+P\+L\+A\+B\+X project}\label{building_building_pic32}
M\+P\+L\+A\+B\+X project resides in the {\ttfamily src/arch/pic32/tneokernel.\+X} directory. This is a {\itshape library project} in terms of M\+P\+L\+A\+B\+X, so if you use M\+P\+L\+A\+B\+X you can easily add it to your main project by right-\/clicking {\ttfamily Libraries -\/$>$ Add Library Project ...}. Alternatively, of course you can just build it and use resulting {\ttfamily tneokernel.\+X.\+a} file in whatever way you like. 