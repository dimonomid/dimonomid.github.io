P\+I\+C32 port implementation details\hypertarget{pic32_details_pic32_context_switch}{}\section{Context switch}\label{pic32_details_pic32_context_switch}
The context switch is implemented using the core software 0 interrupt. It should be configured to use the lowest priority in the system\+:


\begin{DoxyCode}
\textcolor{comment}{// set up the software interrupt 0 with a priority of 1, subpriority 0}
INTSetVectorPriority(INT\_CORE\_SOFTWARE\_0\_VECTOR, INT\_PRIORITY\_LEVEL\_1);
INTSetVectorSubPriority(INT\_CORE\_SOFTWARE\_0\_VECTOR, INT\_SUB\_PRIORITY\_LEVEL\_0);
INTEnable(INT\_CS0, INT\_ENABLED);
\end{DoxyCode}


The interrupt priority level used by the context switch interrupt should not be configured to use shadow register sets.

\begin{DoxyAttention}{Attention}
if tneokernel is built as a separate library, then the file {\ttfamily src/arch/pic32/tn\+\_\+arch\+\_\+pic32\+\_\+int\+\_\+vec1.\+S} must be included in the main project itself, in order to dispatch vector1 (core software interrupt 0) correctly. Do note that if we include this file in the T\+Neo\+Kernel library project, it doesn't work for vector, unfortunately.

If you forgot to include this file, you got an error on the link step, like this\+: 
\begin{DoxyCode}
undefined reference to `\_you\_should\_add\_file\_\_\_tn\_arch\_pic32\_int\_vec1\_S\_\_\_to\_the\_project\textcolor{stringliteral}{'}
\end{DoxyCode}
 Which is much more informative than if you just get to {\ttfamily \+\_\+\+Default\+Interrupt} when it's time to switch context.
\end{DoxyAttention}
\hypertarget{pic32_details_pic32_interrupts}{}\section{Interrupts}\label{pic32_details_pic32_interrupts}
For detailed information about interrupts in T\+Neo\+Kernel, refer to the page \hyperlink{interrupts}{Interrupts}.

P\+I\+C32 port supports nested interrupts. The kernel provides C-\/language macros for calling C-\/language interrupt service routines, which can use either M\+I\+P\+S32 or M\+I\+P\+S16e mode. Both software and shadow register interrupt context saving is supported. Usage is as follows\+:


\begin{DoxyCode}
\textcolor{comment}{/* Timer 1 interrupt handler using software interrupt context saving */}
\hyperlink{tn__arch__pic32_8h_a46a860d030e59e2c6aa827ca5ad36a37}{tn\_soft\_isr}(\_TIMER\_1\_VECTOR)
\{
   \textcolor{comment}{/* here is your ISR code, including clearing of interrupt flag, and so on */}
\}

\textcolor{comment}{/* High-priority UART interrupt handler using shadow register set */}
\hyperlink{tn__arch__pic32_8h_acb97f377bb993c30f6c83c582978d895}{tn\_srs\_isr}(\_UART\_1\_VECTOR)
\{
   \textcolor{comment}{/* here is your ISR code, including clearing of interrupt flag, and so on */}
\}
\end{DoxyCode}


Alternatively, you can define your I\+S\+R in a standard way, like this\+:


\begin{DoxyCode}
\textcolor{keywordtype}{void} \_\_ISR(\_TIMER\_1\_VECTOR) timer\_1\_isr(\textcolor{keywordtype}{void})
\{
   \textcolor{comment}{/* here is your ISR code, including clearing of interrupt flag, and so on */}
\}
\end{DoxyCode}


Then, context is saved on the task's stack instead of interrupt stack (and takes therefore much more R\+A\+M), but you save about 20 cycles for each interrupt. See the page \hyperlink{interrupts}{Interrupts} for details. 