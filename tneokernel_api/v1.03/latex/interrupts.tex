\hypertarget{interrupts_interrupt_stack}{}\section{Interrupt stack}\label{interrupts_interrupt_stack}
T\+Neo\+Kernel provides a separate stack for interrupt handlers. This approach could save a lot of R\+A\+M\+: interrupt can happen at any moment of time, and if there's no separate interrupt stack, then each task should have enough stack space for the worse case of interrupt nesting.

Assume application's I\+S\+Rs take max 64 words (64 $\ast$ 4 = 256 bytes on P\+I\+C32) and application has 4 tasks (plus one idle task). Then, each of 5 tasks must have 64 words for interrupts\+: 64 $\ast$ 5 $\ast$ 4 = 1280 bytes of R\+A\+M just for 64 words for I\+S\+R.

With separate stack for interrupts, these 64 words should be allocated just once. Interrupt stack array should be given to {\ttfamily \hyperlink{tn__sys_8h_a62ab25d9d8ca01c02d368968f19e49bf}{tn\+\_\+sys\+\_\+start()}}. For additional information, refer to the section \hyperlink{quick_guide_starting_the_kernel}{Starting the kernel}.

In order to make particular I\+S\+R use separate interrupt stack, this I\+S\+R should be defined by kernel-\/provided macro, which is platform-\/dependent\+: see \hyperlink{pic32_details}{P\+I\+C32 details}.

In spite of the fact that the kernel provides separate stack for interrupt, this isn't a mandatory\+: you're able to define your I\+S\+R in a standard way, making it use stask of interrupted task and work a bit faster. There is always a tradeoff. There are {\bfseries no additional constraints} on I\+S\+R defined without kernel-\/provided macro\+: in either I\+S\+R, you can call the same set of kernel services.

When you make a decision on whether particular I\+S\+R should use separate stack, consider the following\+:


\begin{DoxyItemize}
\item When I\+S\+R is defined in a standard way, and no function is called from that I\+S\+R, only necessary registers are saved on stack. If you have such an I\+S\+R (that doesn't call any function), and this I\+S\+R should work very fast, consider using standard way instead of kernel-\/provided macro.
\item When I\+S\+R is defined in a standard way, but it calls any function and doesn't use shadow register set, compiler saves (almost) full context {\bfseries on the task's stack}, because it doesn't know which registers are used inside the function. In this case, it usually makes more sense to use kernel-\/provided macro (see below).
\item Kernel-\/provided interrupt macros switch stack pointer between interrupt stack and task stack automatically, it takes additional time\+: e.\+g. on P\+I\+C32 it's about 20 cycles.
\item Kernel-\/provided interrupt macro that doesn't use shadow register set always saves (almost) full context {\bfseries on the interrupt stack}, independently of whether any function is called from an I\+S\+R.
\item Kernel-\/provided interrupt macro that uses shadow register set saves a little amount of registers {\bfseries on the interrupt stack}. 
\end{DoxyItemize}