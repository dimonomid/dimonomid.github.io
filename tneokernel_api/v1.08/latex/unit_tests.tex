Brief information on the implementation of unit tests\hypertarget{unit_tests_tested_cpus}{}\section{Tested C\+P\+Us}\label{unit_tests_tested_cpus}
Currently, unit tests project is tested in the hardware on the following C\+P\+Us\+:


\begin{DoxyItemize}
\item P\+I\+C32\+M\+X440\+F512H
\item P\+I\+C24\+F\+J256\+G\+B106
\end{DoxyItemize}\hypertarget{unit_tests_how_tests_are_implemented}{}\section{How tests are implemented}\label{unit_tests_how_tests_are_implemented}
Briefly\+: there is a high-\/priority task like \char`\"{}test director\char`\"{}, which creates worker tasks as well as various kernel objects (queues, mutexes, etc), and then orders to workers, like\+:


\begin{DoxyItemize}
\item Task A, you lock the mutex M1
\item Task B, you lock the mutex M1
\item Task C, you lock the mutex M1
\item Task A, you delete the mutex M1
\end{DoxyItemize}

After each step it waits for workers to complete their job, and then checks if things are as expected\+: task states, task priorities, last return values of services, various properties of objects, etc.

Detailed log is written to the U\+A\+RT. Typically, for each step, the following is written\+:


\begin{DoxyItemize}
\item verbatim comment is written,
\item director writes what does it do,
\item each worker writes what does it do,
\item director checks things and writes detailed report.
\end{DoxyItemize}

Of course there is a mechanism for writing such scenarios. Here is a part of code that specifies the sequence with locking and deleting mutex explained above\+:


\begin{DoxyCode}
TNT\_TEST\_COMMENT(\textcolor{stringliteral}{"A locks M1"});
TNT\_ITEM\_\_SEND\_CMD\_MUTEX(TNT\_TASK\_\_A, MUTEX\_LOCK, TNT\_MUTEX\_\_1);
TNT\_ITEM\_\_WAIT\_AND\_CHECK\_DIFF(
      TNT\_CHECK\_\_MUTEX(TNT\_MUTEX\_\_1, HOLDER, TNT\_TASK\_\_A);
      TNT\_CHECK\_\_MUTEX(TNT\_MUTEX\_\_1, LOCK\_CNT, 1);

      TNT\_CHECK\_\_TASK(TNT\_TASK\_\_A, LAST\_RETVAL, \hyperlink{tn__common_8h_aa43bd3da1ad4c1e61224b5f23b369876afb291924237186f5765865256c75e639}{TN\_RC\_OK});
      );

TNT\_TEST\_COMMENT(\textcolor{stringliteral}{"B tries to lock M1 -> B blocks, A has priority of B"});
TNT\_ITEM\_\_SEND\_CMD\_MUTEX(TNT\_TASK\_\_B, MUTEX\_LOCK, TNT\_MUTEX\_\_1);
TNT\_ITEM\_\_WAIT\_AND\_CHECK\_DIFF(
      TNT\_CHECK\_\_TASK(TNT\_TASK\_\_B, LAST\_RETVAL, TWORKER\_MAN\_\_LAST\_RETVAL\_\_UNKNOWN);
      TNT\_CHECK\_\_TASK(TNT\_TASK\_\_B, WAIT\_REASON, \hyperlink{tn__oldsymbols_8h_a1d72794d5d13225327301cce148c4676}{TSK\_WAIT\_REASON\_MUTEX\_I});

      TNT\_CHECK\_\_TASK(TNT\_TASK\_\_A, PRIORITY, priority\_task\_b);
      );

TNT\_TEST\_COMMENT(\textcolor{stringliteral}{"C tries to lock M1 -> C blocks, A has priority of C"});
TNT\_ITEM\_\_SEND\_CMD\_MUTEX(TNT\_TASK\_\_C, MUTEX\_LOCK, TNT\_MUTEX\_\_1);
TNT\_ITEM\_\_WAIT\_AND\_CHECK\_DIFF(
      TNT\_CHECK\_\_TASK(TNT\_TASK\_\_C, LAST\_RETVAL, TWORKER\_MAN\_\_LAST\_RETVAL\_\_UNKNOWN);
      TNT\_CHECK\_\_TASK(TNT\_TASK\_\_C, WAIT\_REASON, \hyperlink{tn__oldsymbols_8h_a1d72794d5d13225327301cce148c4676}{TSK\_WAIT\_REASON\_MUTEX\_I});

      TNT\_CHECK\_\_TASK(TNT\_TASK\_\_A, PRIORITY, priority\_task\_c);
      );

TNT\_TEST\_COMMENT(\textcolor{stringliteral}{"A deleted M1 -> B and C become runnable and have retval TN\_RC\_DELETED, A has its base
       priority"});
TNT\_ITEM\_\_SEND\_CMD\_MUTEX(TNT\_TASK\_\_A, MUTEX\_DELETE, TNT\_MUTEX\_\_1);
TNT\_ITEM\_\_WAIT\_AND\_CHECK\_DIFF(
      TNT\_CHECK\_\_TASK(TNT\_TASK\_\_B, LAST\_RETVAL, \hyperlink{tn__common_8h_aa43bd3da1ad4c1e61224b5f23b369876a3252967d2fdefcfbd4eb720ed4663d84}{TN\_RC\_DELETED});
      TNT\_CHECK\_\_TASK(TNT\_TASK\_\_C, LAST\_RETVAL, \hyperlink{tn__common_8h_aa43bd3da1ad4c1e61224b5f23b369876a3252967d2fdefcfbd4eb720ed4663d84}{TN\_RC\_DELETED});
      TNT\_CHECK\_\_TASK(TNT\_TASK\_\_B, WAIT\_REASON, \hyperlink{tn__oldsymbols_8h_a56aeec64ffd044ed5fb917eb415ea889}{TSK\_WAIT\_REASON\_DQUE\_WRECEIVE})
      ;
      TNT\_CHECK\_\_TASK(TNT\_TASK\_\_C, WAIT\_REASON, \hyperlink{tn__oldsymbols_8h_a56aeec64ffd044ed5fb917eb415ea889}{TSK\_WAIT\_REASON\_DQUE\_WRECEIVE})
      ;

      TNT\_CHECK\_\_TASK(TNT\_TASK\_\_A, PRIORITY, priority\_task\_a);

      TNT\_CHECK\_\_MUTEX(TNT\_MUTEX\_\_1, HOLDER, TNT\_TASK\_\_NONE);
      TNT\_CHECK\_\_MUTEX(TNT\_MUTEX\_\_1, LOCK\_CNT, 0);
      TNT\_CHECK\_\_MUTEX(TNT\_MUTEX\_\_1, EXISTS, 0);
      );
\end{DoxyCode}


And here is the appropriate part of log that is echoed to the U\+A\+RT\+:


\begin{DoxyCode}
\textcolor{comment}{//-- A locks M1 (line 404 in ../source/appl/appl\_tntest/appl\_tntest\_mutex.c) }
[I]: tnt\_item\_proceed():2101: ----- Command to task A: lock mutex M1 (0xa0004c40) 
[I]: tnt\_item\_proceed():2160: Wait 80 ticks 
[I]: [Task A]: locking mutex (0xa0004c40).. 
[I]: [Task A]: mutex (0xa0004c40) locked 
[I]: [Task A]: waiting for command.. 
[I]: tnt\_item\_proceed():2178: Checking: 
[I]: * Task A: priority=6 (as expected), wait\_reason=DQUE\_WRECEIVE (as expected), last\_retval=
      \hyperlink{tn__common_8h_aa43bd3da1ad4c1e61224b5f23b369876afb291924237186f5765865256c75e639}{TN\_RC\_OK} (as expected) 
[I]: * Task B: priority=5 (as expected), wait\_reason=DQUE\_WRECEIVE (as expected), last\_retval=NOT-YET-
      RECEIVED (as expected) 
[I]: * Task C: priority=4 (as expected), wait\_reason=DQUE\_WRECEIVE (as expected), last\_retval=NOT-YET-
      RECEIVED (as expected) 
[I]: * Mutex M1: holder=A (as expected), lock\_cnt=1 (as expected), exists=yes (as expected) 

\textcolor{comment}{//-- B tries to lock M1 -> B blocks, A has priority of B (line 413 in
       ../source/appl/appl\_tntest/appl\_tntest\_mutex.c) }
[I]: tnt\_item\_proceed():2101: ----- Command to task B: lock mutex M1 (0xa0004c40) 
[I]: tnt\_item\_proceed():2160: Wait 80 ticks 
[I]: [Task B]: locking mutex (0xa0004c40).. 
[I]: tnt\_item\_proceed():2178: Checking: 
[I]: * Task A: priority=5 (as expected), wait\_reason=DQUE\_WRECEIVE (as expected), last\_retval=
      \hyperlink{tn__common_8h_aa43bd3da1ad4c1e61224b5f23b369876afb291924237186f5765865256c75e639}{TN\_RC\_OK} (as expected) 
[I]: * Task B: priority=5 (as expected), wait\_reason=MUTEX\_I (as expected), last\_retval=NOT-YET-RECEIVED (
      as expected) 
[I]: * Task C: priority=4 (as expected), wait\_reason=DQUE\_WRECEIVE (as expected), last\_retval=NOT-YET-
      RECEIVED (as expected) 
[I]: * Mutex M1: holder=A (as expected), lock\_cnt=1 (as expected), exists=yes (as expected) 

\textcolor{comment}{//-- C tries to lock M1 -> B blocks, A has priority of C (line 422 in
       ../source/appl/appl\_tntest/appl\_tntest\_mutex.c) }
[I]: tnt\_item\_proceed():2101: ----- Command to task C: lock mutex M1 (0xa0004c40) 
[I]: tnt\_item\_proceed():2160: Wait 80 ticks 
[I]: [Task C]: locking mutex (0xa0004c40).. 
[I]: tnt\_item\_proceed():2178: Checking: 
[I]: * Task A: priority=4 (as expected), wait\_reason=DQUE\_WRECEIVE (as expected), last\_retval=
      \hyperlink{tn__common_8h_aa43bd3da1ad4c1e61224b5f23b369876afb291924237186f5765865256c75e639}{TN\_RC\_OK} (as expected) 
[I]: * Task B: priority=5 (as expected), wait\_reason=MUTEX\_I (as expected), last\_retval=NOT-YET-RECEIVED (
      as expected) 
[I]: * Task C: priority=4 (as expected), wait\_reason=MUTEX\_I (as expected), last\_retval=NOT-YET-RECEIVED (
      as expected) 
[I]: * Mutex M1: holder=A (as expected), lock\_cnt=1 (as expected), exists=yes (as expected) 

\textcolor{comment}{//-- A deleted M1 -> B and C become runnable and have retval TN\_RC\_DELETED, A has its base priority (line
       431 in ../source/appl/appl\_tntest/appl\_tntest\_mutex.c) }
[I]: tnt\_item\_proceed():2101: ----- Command to task A: delete mutex M1 (0xa0004c40) 
[I]: tnt\_item\_proceed():2160: Wait 80 ticks 
[I]: [Task A]: deleting mutex (0xa0004c40).. 
[I]: [Task C]: mutex (0xa0004c40) locking failed with err=-8 
[I]: [Task C]: waiting for command.. 
[I]: [Task B]: mutex (0xa0004c40) locking failed with err=-8 
[I]: [Task B]: waiting for command.. 
[I]: [Task A]: mutex (0xa0004c40) deleted 
[I]: [Task A]: waiting for command.. 
[I]: tnt\_item\_proceed():2178: Checking: 
[I]: * Task A: priority=6 (as expected), wait\_reason=DQUE\_WRECEIVE (as expected), last\_retval=
      \hyperlink{tn__common_8h_aa43bd3da1ad4c1e61224b5f23b369876afb291924237186f5765865256c75e639}{TN\_RC\_OK} (as expected) 
[I]: * Task B: priority=5 (as expected), wait\_reason=DQUE\_WRECEIVE (as expected), last\_retval=
      \hyperlink{tn__common_8h_aa43bd3da1ad4c1e61224b5f23b369876a3252967d2fdefcfbd4eb720ed4663d84}{TN\_RC\_DELETED} (as expected) 
[I]: * Task C: priority=4 (as expected), wait\_reason=DQUE\_WRECEIVE (as expected), last\_retval=
      \hyperlink{tn__common_8h_aa43bd3da1ad4c1e61224b5f23b369876a3252967d2fdefcfbd4eb720ed4663d84}{TN\_RC\_DELETED} (as expected) 
[I]: * Mutex M1: holder=NONE (as expected), lock\_cnt=0 (as expected), exists=no (as expected) 
\end{DoxyCode}


If something goes wrong, there would be no {\ttfamily \char`\"{}as expected\char`\"{}}, but error and explanation what we expected and what we have. Tests halted.

I do my best to model nearly all possible situations within the each single subsystem (such as mutexes, queues, etc), including various situations with suspended tasks, deleted tasks, deleted objects, and the like. It helps a lot to keep the kernel really stable.\hypertarget{unit_tests_get_tests}{}\section{Get unit-\/tests}\label{unit_tests_get_tests}
Currently, there is a separate repository with unit tests for T\+Neo.

Please note that code of unit tests project is not as polished as the code of the kernel itself. My open-\/source time is limited, and I prefer to invest it in the kernel as much as possible.

Nevertheless, unit tests do their job efficiently, which is needed.

There is an \char`\"{}environment\char`\"{} repository, which contains tests and all the necessary library subrepos\+: \href{http://hg.dfrank.ru/tntest/_env}{\tt http\+://hg.\+dfrank.\+ru/tntest/\+\_\+env}

You can clone it as follows\+: \begin{DoxyVerb}hg clone http://hg.dfrank.ru/tntest/_env tntest
\end{DoxyVerb}


The single repository with the tests resides here\+: \href{http://hg.dfrank.ru/tntest/project_common}{\tt http\+://hg.\+dfrank.\+ru/tntest/project\+\_\+common} 