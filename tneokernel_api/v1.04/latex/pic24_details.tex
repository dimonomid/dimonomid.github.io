P\+I\+C24/ds\+P\+I\+C port implementation details\hypertarget{pic24_details_pic24_context_switch}{}\section{Context switch}\label{pic24_details_pic24_context_switch}
The context switch is implemented using the external interrupt 0 ({\ttfamily I\+N\+T0}). It is handled completely by the kernel, application should never touch it.\hypertarget{pic24_details_pic24_interrupts}{}\section{Interrupts}\label{pic24_details_pic24_interrupts}
For detailed information about interrupts in T\+Neo\+Kernel, refer to the page \hyperlink{interrupts}{Interrupts}.

P\+I\+C24/ds\+P\+I\+C T\+Neo\+Kernel port supports nested interrupts. It allows to specify the range of {\itshape system interrupt priorities}. Refer to the section \hyperlink{interrupts_interrupt_types}{Interrupt types} for details on what is {\itshape system interrupt}.

System interrupts use separate interrupt stack instead of the task's stack. This approach saves a lot of R\+A\+M.

The range is specified by just a single number\+: {\ttfamily \hyperlink{tn__cfg__default_8h_a4feb7eb34fc2f175167b7496b63c398a}{T\+N\+\_\+\+P24\+\_\+\+S\+Y\+S\+\_\+\+I\+P\+L}}, which represents maximum {\itshape system interrupt priority}. Here is a list of available priorities and their characteristics\+:


\begin{DoxyItemize}
\item priorities {\ttfamily \mbox{[}1 .. \hyperlink{tn__cfg__default_8h_a4feb7eb34fc2f175167b7496b63c398a}{T\+N\+\_\+\+P24\+\_\+\+S\+Y\+S\+\_\+\+I\+P\+L}\mbox{]}}\+:
\begin{DoxyItemize}
\item Kernel services {\bfseries are} allowed to call;
\item The macro {\ttfamily \hyperlink{tn__arch__pic24_8h_a0b184d3c15066f5504144379d2624ff3}{tn\+\_\+p24\+\_\+soft\+\_\+isr()}} {\bfseries must} be used.
\item Separate interrupt stack is used by I\+S\+R;
\item Interrupts get disabled for short periods of time when modifying critical kernel data (for about 100 cycles or the like).
\end{DoxyItemize}
\item priorities {\ttfamily \mbox{[}(\hyperlink{tn__cfg__default_8h_a4feb7eb34fc2f175167b7496b63c398a}{T\+N\+\_\+\+P24\+\_\+\+S\+Y\+S\+\_\+\+I\+P\+L} + 1) .. 6\mbox{]}}\+:
\begin{DoxyItemize}
\item Kernel services {\bfseries are not} allowed to call;
\item The macro {\ttfamily \hyperlink{tn__arch__pic24_8h_a0b184d3c15066f5504144379d2624ff3}{tn\+\_\+p24\+\_\+soft\+\_\+isr()}} {\bfseries must not} be used.
\item Task's stack is used by I\+S\+R;
\item Interrupts are not disabled when modifying critical kernel data, but they are disabled for 4..8 cycles by {\ttfamily disi} instruction when entering/exiting system I\+S\+R\+: we need to safely modify {\ttfamily S\+P} and {\ttfamily S\+P\+L\+I\+M}.
\end{DoxyItemize}
\item priority {\ttfamily 7}\+:
\begin{DoxyItemize}
\item Kernel services {\bfseries are not} allowed to call;
\item The macro {\ttfamily \hyperlink{tn__arch__pic24_8h_a0b184d3c15066f5504144379d2624ff3}{tn\+\_\+p24\+\_\+soft\+\_\+isr()}} {\bfseries must not} be used.
\item Task's stack is used by I\+S\+R;
\item Interrupts are never disabled by the kernel. Note that {\ttfamily disi} instruction leaves interrupts of priority 7 enabled.
\end{DoxyItemize}
\end{DoxyItemize}

The kernel provides C-\/language macro for calling C-\/language {\bfseries system} interrupt service routines.

Usage is as follows\+:


\begin{DoxyCode}
\textcolor{comment}{/* }
\textcolor{comment}{ * Timer 1 interrupt handler using software interrupt context saving,  }
\textcolor{comment}{ * PSV is handled automatically:}
\textcolor{comment}{ */}
\hyperlink{tn__arch__pic24_8h_a0b184d3c15066f5504144379d2624ff3}{tn\_p24\_soft\_isr}(\_T1Interrupt, auto\_psv)
\{
   \textcolor{comment}{//-- clear interrupt flag}
   IFS0bits.T1IF = 0;

   \textcolor{comment}{//-- do something useful}
\}
\end{DoxyCode}


\begin{DoxyAttention}{Attention}
do {\bfseries not} use this macro for non-\/system interrupt (that is, for interrupt of priority higher than {\ttfamily \hyperlink{tn__cfg__default_8h_a4feb7eb34fc2f175167b7496b63c398a}{T\+N\+\_\+\+P24\+\_\+\+S\+Y\+S\+\_\+\+I\+P\+L}}). Use standard way to define it. If you violate this rule, debugger will be halted by the kernel when entering I\+S\+R. In release build, C\+P\+U is just reset.
\end{DoxyAttention}
\hypertarget{pic24_details_pic24_bfa}{}\section{Atomic access to the structure bit field}\label{pic24_details_pic24_bfa}
The problem with P\+I\+C24/ds\+P\+I\+C is that when we write something like\+:


\begin{DoxyCode}
IPC0bits.INT0IP = 0x05;
\end{DoxyCode}


We actually have read-\/modify-\/write sequence which can be interrupted, so that resulting data could be corrupted. P\+I\+C24/ds\+P\+I\+C port provides several macros that offer atomic access to the structure bit field.

The kernel would not probably provide that kind of functionality, but T\+Neo\+Kernel itself needs it, so, it is made public so that application can use it too.

Refer to the page \hyperlink{tn__arch__pic24__bfa_8h}{Atomic bit-\/field access macros} for details.\hypertarget{pic24_details_pic24_building}{}\section{Building}\label{pic24_details_pic24_building}
For generic information on building T\+Neo\+Kernel, refer to the page \hyperlink{building}{Building the project}.

M\+P\+L\+A\+B\+X project for P\+I\+C24/ds\+P\+I\+C port resides in the {\ttfamily src/arch/pic24\+\_\+dspic/tneokernel\+\_\+pic24\+\_\+dspic.\+X} directory. This is a {\itshape library project} in terms of M\+P\+L\+A\+B\+X, so if you use M\+P\+L\+A\+B\+X you can easily add it to your main project by right-\/clicking {\ttfamily Libraries -\/$>$ Add Library Project ...}.

\begin{DoxyAttention}{Attention}
there are two configurations of this project\+: {\itshape eds} and {\itshape  no\+\_\+eds}, for devices with and without extended data space, respectively. When you add library project to your application project, you should select correct configuration for your device; otherwise, you get \char`\"{}undefined reference\char`\"{} errors at linker step.
\end{DoxyAttention}
Alternatively, of course you can just build it and use resulting {\ttfamily .a} file in whatever way you like. 